\documentclass[a4paper, 12pt]{article}

\begin{document}

\begin{titlepage}
	\begin{center}
		\line(1,0){300} \\
		\huge{\bfseries Tarea 1 }
		\line(1,0){300} \\
		\line(1,0){300} \\
		Sistemas Operativos \\
		{\normalsize ING. Gunnar Eyal Wolf Iszaevich }\\
		{\normalsize Ricardo Rosales Romero }\\
		{\small 12 Agosto 2019}
	\end{center}
\end{titlepage}

\section {Jefe - Trabajador}

\subsection*{An\'alisis del problema}


{\bf  Breve descripción}\\ \\

Utilizando lenguaje de Programaci\'on Python e importando las librer\'ias de Threading , Time y Random 
\\ \\
Utilicé un m\'ultiplex para impedir que a varios trabajadores se les asignara la misma p\'agina a buscar
\\ \\
El refinamiento que le agregu\'e es que devuelva la p\'agina que encontr\'o, qu\'e avise cuando ya acab\'o y  el tiempo que se tard\'o.
\\ \\
No esto seguro  que fuera lo que esperaba usted que entregar\'a . Ya que se me hicieron muy pocos los sem\'aforos que use a comparaci\'on de varios ejemplos vistos de forma previa. Quiz\'a pude evitar conflictos que se pudieran encontrar en el futuro de vida del programa. Y sin duda no me he puesto a analizarlos todos. Quiz\'a con la pr\'actica mejore mi capacidad de interpretar  el multi-procesamiento y el manejo de todos los errores que mencionamos en clase. Cabe resaltar que la condici\'on de carrera y la concurrencia se controlan en mi c\'odigo, seg\'un yo. 
\\ \\

\end{document}